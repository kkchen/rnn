\section{Alternate architectures}
\subsection{}

% To do: explain HMM.
\begin{frame}{RNNs and Hidden Markov Models}
    \begin{columns}
        \begin{column}{0.46\textwidth}
            \begin{tikzpicture}[x=9mm, y=7mm]
    % All States.
    \foreach \i in {1, ..., 5, 7} {
        \foreach \j in {1, 3, 4, ..., 6} {
            \node (\i\j) [hmm] at (\i, \j) {};
        }

        \node (\i2) at (\i, 2) [label] {\vvdots};
    }

    \foreach \j in {1, 3, 4, ..., 6} {
        \node (6\j) at (6, \j) [label] {$\cdots$};
    }

    \node (62) [circle, minimum width=2mm, inner sep=0pt] at (6, 2) {};

    % Traversed states.
    \node (state 13) [hmm, fill=blue] at (1, 3) {};
    \node (state 21) [hmm, fill=blue] at (2, 1) {};
    \node (state 33) [hmm, fill=blue] at (3, 3) {};
    \node (state 45) [hmm, fill=blue] at (4, 5) {};
    \node (state 56) [hmm, fill=blue] at (5, 6) {};
    \node (state 74) [hmm, fill=blue] at (7, 4) {};

    % All connections.
    \foreach \i in {1, ..., 6} {
        \pgfmathtruncatemacro{\j}{\i + 1}

        \foreach \k in {1, ..., 6} {
            \foreach \l in {1, ..., 6} {
                \draw [black!30] (\i\k) -- (\j\l);
            }
        }
    }

    % Traversed connections.
    \draw [blue, thick] (state 13) -- (state 21);
    \draw [blue, thick] (state 21) -- (state 33);
    \draw [blue, thick] (state 33) -- (state 45);
    \draw [blue, thick] (state 45) -- (state 56);
    \draw [blue, thick] (state 56) -- (62);
    \draw [blue, thick] (62) -- (state 74);

    % Outputs.
    \draw [thick] (0.9, 0.25) -- (7.1, 0.25);

    \foreach \i in {1, ..., 5, 7} {
        \node [hmm, fill=red] at (\i, -0.5) {};
    }

    \node at (6, -0.5) {$\cdots$};

    \foreach \i in {1, ..., 7} {
        \draw [path, blue] (\i, 0.7) -- (\i, -0.3);
    }
\end{tikzpicture}

%%% Local Variables:
%%% mode: latex
%%% TeX-master: "../rnn"
%%% End:

        \end{column}
        \begin{column}{0.54\textwidth}
            \begin{itemize}
                \item<+-> RNNs are deterministic, HMMs are stochastic
                \item<.-> But posterior probabilities of states and outputs of HMMs are deterministic
            \end{itemize}
            \begin{block}{}<+->
                An HMM, when viewed in terms of probabilities, is a special case of an RNN
            \end{block}
            \begin{itemize}[<.->]
                \item HMM transition probabilities analogous to RNN weights
                \item<+-> Why do traditional RNNs and HMMs excel at different tasks?
                \item What about a task makes each work well?
                \item What can each do well that the other cannot?
            \end{itemize}
        \end{column}
    \end{columns}
\end{frame}

\begin{frame}{1-D convolutional neural networks}
    Sequences analogous to 1-D space
    \begin{itemize}
        \item<+-> Physics/engineering analogy: time analogous to 1-D space
        \item ODE methods translate between the two
    \end{itemize}
    \uncover<+->{Can use 1-D CNNs to process sequences}

    \vspace{-4mm}
    \begin{center}
        \begin{tikzpicture}[x=4mm, y=5mm]
    \node (input 0) [coordinate] at (0, 0) {};
    \node (output 0) [coordinate] at (0, 2.2) {};

    \uncover<.->{
        \foreach \i in {0, ..., 20} {
            \fill [black] (\i, 0) circle (0.7mm);
        }

        \node [align=right, left=3mm of input 0] {input};
        \draw [path] (8, -0.5) -- node [label, below] {$t$} (12, -0.5);
    }

    \visible<3->{
        \foreach \i in {0, 1} {
            \fill[blue!30] (\i, 2.2) circle (0.7mm);
        }

        \node [align=right, left=3mm of output 0, blue] {output};
    }

    \foreach \i in {2, ..., 6} {
        \visible<+|handout:0>{
            \node [label, blue] at (\i, 0.6) {convolve};

            \foreach \j in {-2, ..., 2} {
                \fill [red] (\i+\j, 1.1) circle (0.7mm);
            }

            \draw [thick, blue] (\i-2.4, 0) -- (\i-2.4, 1.4) -- (\i+2.4, 1.4) -- (\i+2.4, 0);
            \node at (\i+2, 1.1) [label, anchor=west, xshift=2mm, red] {kernel};
            \draw [-Latex, blue] (\i, 1.4) -- (\i, 2);
        }

        \visible<.->{\fill [blue] (\i, 2.2) circle (0.7mm);}
    }

    \visible<+->{
        \def\i{18}
        \node [label, blue] at (\i, 0.6) {convolve};

        \foreach \j in {-2, ..., 2} {
            \fill [red] (\i+\j, 1.1) circle (0.7mm);
        }

        \draw [thick, blue] (\i-2.4, 0) -- (\i-2.4, 1.4) -- (\i+2.4, 1.4) -- (\i+2.4, 0);
        \node at (\i+2, 1.1) [label, anchor=west, xshift=2mm, red] {kernel};
        \draw [-Latex, blue] (\i, 1.4) -- (\i, 2);
        \fill [blue] (\i, 2.2) circle (0.7mm);

        \foreach \i in {7, ..., 17} {
            \fill [blue] (\i, 2.2) circle (0.7mm);
        }

        \foreach \i in {19, 20} {
            \fill [blue!30] (\i, 2.2) circle (0.7mm);
        }
    }
\end{tikzpicture}

%%% Local Variables:
%%% mode: latex
%%% TeX-master: "../rnn"
%%% End:

        \vspace{4mm}

        \uncover<+->{
            \begin{tabular}{c|c}
                RNN & CNN \\
                \hline
                $\y_t$ depends on $\s_0, \dots, \s_t$ &
                $\y_t$ depends on $\s_{t-\delta}, \dots, \s_{t+\delta}$ \\
                (also $\s_{t+1}, \dots, \s_\tau$ if desired) \\
                generally non-parallelizable across $t$ & parallelizable across $t$ \\
            \end{tabular}
        }
    \end{center}

    \uncover<.->{
        RNNs generally stronger than 1-D CNNs for sequences;
        RNN cell state intuitively helpful
    }
\end{frame}

\begin{frame}{Recursive neural networks}
    \begin{tikzpicture}[x=1.2cm, y=1.3cm, auto]
    \foreach \i in {1, ..., 8} {
        \node (input \i) [state] at (\i, 0) {$\x_\i$};
        \node (state 1\i) [state] at (\i, 1) {};
        \draw [path] (input \i) -- node [label, right, pos=0.4] {$\f$} (state 1\i);
    }

    \node (state 21) [state] at (1.5, 2) {};
    \node (state 22) [state] at (3.5, 2) {};
    \node (state 23) [state] at (5.5, 2) {};
    \node (state 24) [state] at (7.5, 2) {};

    \node (state 31) [state] at (2.5, 3) {};
    \node (state 32) [state] at (6.5, 3) {};

    \node (output) [state] at (4.5, 4) {$\y$};
    \node (loss) [function] at (4.5, 5) {$L$};
    \node (data) [data] at (3.5, 5) {$\ETA$};

    \foreach \j in {1, ..., 4} {
        \pgfmathtruncatemacro{\i}{2 * \j - 1}
        \draw [path] (state 1\i) -- node [word, label, left] {$\g$} (state 2\j);
    }

    \foreach \j in {1, ..., 4} {
        \pgfmathtruncatemacro{\i}{2 * \j}
        \draw [path] (state 1\i) -- node [label, right] {$\h$} (state 2\j);
    }

    \foreach \i/\j in {1/1, 3/2} {
        \draw [path] (state 2\i) -- node [word, label, above, pos=0.1] {$\g$} (state 3\j);
    }

    \foreach \i/\j in {2/1, 4/2} {
        \draw [path] (state 2\i) -- node [label, above, pos=0.1] {$\h$} (state 3\j);
    }

    \draw [path] (state 31) -- node [word, label, above, pos=0.4] {$\g$} (output);
    \draw [path] (state 32) -- node [word, label, above, pos=0.4] {$\h$} (output);

    \draw [path] (output) -- (loss);
    \draw [path] (data) -- (loss);
\end{tikzpicture}

%%% Local Variables:
%%% mode: latex
%%% TeX-master: "../rnn"
%%% End:


    \begin{textblock}{5.1}(6.7, 1.3)
        \begin{itemize}
            \item Introduced by \citet{PollackAI90}
            \item Advantage: longest connection $\approx \log_2 \tau$, not $\tau$
            \item Tree structure can be static or data-dependent
        \end{itemize}
    \end{textblock}
\end{frame}

%%% Local Variables:
%%% mode: latex
%%% TeX-master: "../rnn"
%%% End:
