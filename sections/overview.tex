\section{Overview}
\subsection{}

\begin{frame}{Resources}
    \begin{itemize}
        \item Ian Goodfellow, Yoshua Bengio, and Aaron Courville.
        \emph{Deep Learning}, ch.~10. MIT Press, 2016
        \nocite{GoodfellowDL}
        \begin{itemize}
            \item Excellent primer
            \item \textcolor{blue}{\url{http://www.deeplearningbook.org}}
        \end{itemize}
        \item References at the end of these slides
        \item Talk to me! \smiley
    \end{itemize}
\end{frame}

\begin{frame}{Why might I want an \rnn?}
    \begin{itemize}
        \item Universal approximation theorem: dense \nn{}s can model anything nice
        \begin{itemize}
            \item But doesn't specify how hard it is to train
        \end{itemize}
        \item Silly example: 5 seconds of 44,100 samples/s speech audio $\rightarrow$ 14 transcribed words
        \begin{itemize}
            \item $\exists$ $\sim$3000 commonly-used English words\footnote{The Internet}
            \item $\Reals^{220,500} \to \Reals^{42,000}$ deep dense network?
            (Theoretically possible, but please don't)
            \item What if I speak 15 words over 5.01~s?
        \end{itemize}
    \end{itemize}
    \pause

    \begin{block}{}
        Many processes are sequences through \alert{time} or \alert{steps}, with \alert{time-invariant dynamics}
    \end{block}
    \pause

    \begin{itemize}
        \item \rnn{}s are input--output models that force this structure
        \begin{itemize}
            \item E.g., \rnn{}s treat sample 213,913 same as 213,914; dense \nn{}s treat them entirely separately
        \end{itemize}
        \item Easier to converge to model with ``reasonable''/``desirable'' behavior
        \item Far fewer parameters than naive dense networks
        \item Flexible in how many steps to take
    \end{itemize}
\end{frame}

\begin{frame}{Unfolded computational graphs (left) \& circuit diagrams (right)}
    \begin{columns}
        \begin{column}{0.71\textwidth}
            \uncover<+->{\begin{tikzpicture}[node distance=7mm, auto]
    % States.
    \node (state0) [state] {$\s_0$};
    \node (state1) [state, right=of state0] {$\s_1$};
    \node (state2) [state, right=of state1] {$\s_2$};
    \node (state3) [block, right=of state2] {$\cdots$};
    \node (state4) [state, right=of state3] {$\s_\tau$};

    % Inputs, outputs, losses.
    \foreach\i in {1,2} {
        \node (input\i) [state, below=of state\i] {$\x_\i$};
        \node (output\i) [state, above=of state\i] {$\y_\i$};
        \node (loss\i) [function, above=of output\i] {$L$};
        \node (data\i) [data, above=of loss\i] {$\ETA_\i$};
    }

    \node (input3) [block, below=of state3] {$\cdots$};
    \node (output3) [block, above=of state3] {$\cdots$};
    \node (loss3) [block, above=of output3] {$\cdots$};
    \node (data3) [block, above=of loss3] {$\cdots$};

    \node (input4) [state, below=of state4] {$\x_\tau$};
    \node (output4) [state, above=of state4] {$\y_\tau$};
    \node (loss4) [function, above=of output4] {$L$};
    \node (data4) [data, above=of loss4] {$\ETA_\tau$};

    % Arrows.
    \foreach\i in {1, ..., 4} {
        \draw [path] (input\i) -- node [label, pos=0.4, right] {$\f$} (state\i);
        \draw [path] (state\i) -- node [label, pos=0.4, right] {$\h$} (output\i);
        \draw [path] (output\i) -- (loss\i);
        \draw [path] (data\i) -- (loss\i);
    }

    \foreach\i/\j in {0/1, 1/2, 2/3, 3/4} {
        \draw [path] (state\i) -- node [label, pos=0.4] {$\f$} (state\j);
    }
\end{tikzpicture}
%%% Local Variables:
%%% mode: latex
%%% TeX-master: "../rnn"
%%% End:
}
            \uncover<+->{\begin{tikzpicture}[node distance=7mm, auto]
    \node (state) [state] {$\s_t$};
    \node (input) [state, below=of state] {$\x_t$};
    \node (output) [state, above=of state] {$\y_t$};
    \node (loss) [function, above=of output] {$L$};
    \node (data) [data, above=of loss] {$\ETA_t$};
    \node (skip) [inner sep=0.5mm, above=1mm of state] {};
    \node (skip l) [coordinate, left=7mm of skip] {};
    \node (skip r) [coordinate, right=7mm of skip] {};

    % Arrows.
    \draw [path] (input) -- node [label, pos=0.4, right] {$\f$} (state);
    \draw [path] (state) -- node [label, pos=0.6, right] {$\h$} (output);
    \draw [path] (output) -- (loss);
    \draw [path] (data) -- (loss);
    \draw [thick] (state) -| node [label, right, pos=0.7] {$\f$} (skip r) -- (skip);
    \draw [path] (skip) -- (skip l) |- (state);
\end{tikzpicture}
%%% Local Variables:
%%% mode: latex
%%% TeX-master: "../rnn"
%%% End:
}
        \end{column}
        \begin{column}{0.29\textwidth}
            \begin{itemize}
                \item<1-> Unfolded graph shows flow of vectors through \rnn
                \item<2-> Circuit diagram compresses out time dimension
            \end{itemize}
        \end{column}
    \end{columns}
\end{frame}

\begin{frame}{Recurrent neural networks in one slide}
    \begin{itemize}[<+->]
        \item Express in terms of discrete-time dynamical systems: $t = 0, 1, \dots, \tau$
        \item $\s_t \in \Reals^n$: (cell) state; initial condition $\s_0$ (often $\zero$)
        \item<.-> $\x_t \in \Reals^q$: inputs
        \item $\f: \Reals^n \times \Reals^q \to \Reals^n$: state update
        \item<.-> $\THETA \in \Reals^m$: parameters for $\f$
        \item State update:
        \begin{equation*}
            \s_t = \f(\s_{t-1}, \x_t; \THETA),
            \quad \text{e.g.}, \,
            \s_t = \tanh(\W \s_{t-1} + \U \x_t + \b), \;
            \THETA = (\W, \U, \b)
        \end{equation*}
        \item Outputs: $\y_t \in \Reals^p$, $\h: \Reals^n \to \Reals^p$, parameters $\PHI \in \Reals^l$:
        \begin{equation*}
            \y_t = \h(\s_t; \PHI),
            \quad \text{e.g.}, \,
            \y_t = \mathrm{softmax}(\V \s_{t} + \c), \;
            \PHI = (\V, \c)
        \end{equation*}
        \item That's it!  Now pick $\THETA$ and $\PHI$ to minimize loss $L(
            \underbrace{(\ETA_1, \dots, \ETA_\tau)}_\text{data},
            \underbrace{(\y_1, \dots, \y_\tau)}_\text{prediction}
        )$
    \end{itemize}
\end{frame}

\begin{frame}{Backpropagation \& unrolling}
    \begin{itemize}
        \item<+-> I won't bore you with backprop equations; see \href{http://www.deeplearningbook.org}{\textcolor{blue}{Goodfellow's book}}
        \item<.-> It's just a normal gradient via chain rule; any optimizer can be used
        \item<+-> Need to forward and backprop through input--state--output relations, \emph{and} through time
        \begin{itemize}
            \item Forward and backprop generally serial through time
        \end{itemize}
        \item<.-> For state $\in \Reals^n$, inputs $\in \Reals^q$, outputs $\in \Reals^p$, length $\tau$:
        \begin{itemize}
            \item Backprop cost = $\O(\tau n^2) + \O(\tau n p) + \O(\tau n q)$ products
            \item Watch out for very large \# time steps and input/state/output sizes
        \end{itemize}
        \item<+-> Software typically allows you to \alert{unroll} an \rnn{}
        \begin{itemize}
            \item Precomputes entire unfolded computational graph \& all gradients through all time steps
            \item Faster, but more memory consumption
            \item Good for small, fixed-length \rnn{}s
        \end{itemize}
    \end{itemize}
\end{frame}

%%% Local Variables:
%%% mode: latex
%%% TeX-master: "../rnn"
%%% End:
