\section{Recurrent neural network connections}
\subsection{}

\begin{frame}{State-to-state connections}
    \begin{columns}
        \begin{column}{0.602\textwidth}
            \begin{tikzpicture}[node distance=7mm, auto]
    % States.
    \node (state0) [state] {$\s_0$};
    \node (state1) [state, right=of state0] {$\s_1$};
    \node (state2) [state, right=of state1] {$\s_2$};
    \node (state3) [block, right=of state2] {$\cdots$};
    \node (state4) [state, right=of state3] {$\s_\tau$};

    % Inputs, outputs, losses.
    \foreach\i in {1,2} {
        \node (input\i) [state, below=of state\i] {$\x_\i$};
        \node (output\i) [state, above=of state\i] {$\y_\i$};
        \node (loss\i) [function, above=of output\i] {$L$};
        \node (data\i) [data, above=of loss\i] {$\ETA_\i$};
    }

    \node (input3) [block, below=of state3] {$\cdots$};
    \node (output3) [block, above=of state3] {$\cdots$};
    \node (loss3) [block, above=of output3] {$\cdots$};
    \node (data3) [block, above=of loss3] {$\cdots$};

    \node (input4) [state, below=of state4] {$\x_\tau$};
    \node (output4) [state, above=of state4] {$\y_\tau$};
    \node (loss4) [function, above=of output4] {$L$};
    \node (data4) [data, above=of loss4] {$\ETA_\tau$};

    % Arrows.
    \foreach\i in {1, ..., 4} {
        \draw [path] (input\i) -- node [label, pos=0.4, right] {$\f$} (state\i);
        \draw [path] (state\i) -- node [label, pos=0.4, right] {$\h$} (output\i);
        \draw [path] (output\i) -- (loss\i);
        \draw [path] (data\i) -- (loss\i);
    }

    \foreach\i/\j in {0/1, 1/2, 2/3, 3/4} {
        \draw [path] (state\i) -- node [label, pos=0.4] {$\f$} (state\j);
    }
\end{tikzpicture}
%%% Local Variables:
%%% mode: latex
%%% TeX-master: "../rnn"
%%% End:

        \end{column}
        \begin{column}{0.398\textwidth}
            \begin{itemize}
                \item State-to-state connections are advantageous
                \item Hidden state $\s_t$ (ideally) encodes all information about dynamics up to time $t$, in relatively unconstrained way
                \item Problem: $\s_t$ time-coupling $\implies$ non-parallelizable
            \end{itemize}
        \end{column}
    \end{columns}
\end{frame}

\begin{frame}{Output-to-state connections}
    \begin{columns}
        \begin{column}{0.47\textwidth}
            \begin{tikzpicture}[node distance=7mm, auto]
    % States.
    \node (state1) [state] {$\s_1$};
    \node (state2) [state, right=of state1] {$\s_2$};
    \node (state3) [block, right=of state2] {$\cdots$};
    \node (state4) [state, right=of state3] {$\s_\tau$};

    % Inputs, outputs, losses.
    \foreach\i in {1,2} {
        \node (input\i) [state, below=of state\i] {$\x_\i$};
        \node (output\i) [state, above=of state\i] {$\y_\i$};
        \node (loss\i) [function, above=of output\i] {$L$};
        \node (data\i) [data, above=of loss\i] {$\ETA_\i$};
    }

    \node (input3) [block, below=of state3] {$\cdots$};
    \node (output3) [block, above=of state3] {$\cdots$};
    \node (loss3) [block, above=of output3] {$\cdots$};
    \node (data3) [block, above=of loss3] {$\cdots$};

    \node (input4) [state, below=of state4] {$\x_\tau$};
    \node (output4) [state, above=of state4] {$\y_\tau$};
    \node (loss4) [function, above=of output4] {$L$};
    \node (data4) [data, above=of loss4] {$\ETA_\tau$};

    % Arrows.
    \foreach\i in {1, ..., 4} {
        \draw [path] (input\i) -- node [label, pos=0.4, right] {$\f$} (state\i);
        \draw [path] (state\i) -- node [label, pos=0.4, right] {$\h$} (output\i);
        \draw [path] (output\i) -- (loss\i);
        \draw [path] (data\i) -- (loss\i);
    }

    \foreach\i/\j in {1/2, 2/3, 3/4} {
        \draw [path] (output\i) -- node [label, pos=0.6, above] {$\f$} (state\j);
    }
\end{tikzpicture}
%%% Local Variables:
%%% mode: latex
%%% TeX-master: "../rnn"
%%% End:

        \end{column}
        \begin{column}{0.53\textwidth}
            \begin{itemize}
                \item Weaker: $\y_t$ must encode transition to $\s_{t+1}$ \emph{and} match $\ETA_t$
                \item Forward-prop still non-parallelizable
                \item $\s_t$ not time-coupled
                \item Each time step can be trained independently
                \begin{itemize}
                    \item $\implies$ Backprop parallelizable
                \end{itemize}
                \item Not obvious---rederive backprop equations yourself \smiley
            \end{itemize}
        \end{column}
    \end{columns}
\end{frame}

\begin{frame}{Teacher forcing}
    \begin{columns}
        \begin{column}{0.47\textwidth}
            \begin{tikzpicture}[node distance=7mm, auto]
    % States.
    \node (state0) [draw, block, white] {};
    \node (state1) [state, right=of state0] {$\s_1$};
    \node (state2) [state, right=of state1] {$\s_2$};
    \node (state3) [block, right=of state2] {$\cdots$};
    \node (state4) [state, right=of state3] {$\s_\tau$};

    % Inputs, outputs, losses.
    \foreach\i in {1,2} {
        \node (input\i) [state, below=of state\i] {$\x_\i$};
        \node (output\i) [state, above=of state\i] {$\y_\i$};
        \node (loss\i) [function, above=of output\i] {$L$};
        \node (data\i) [data, above=of loss\i] {$\ETA_\i$};
    }

    \node (input3) [block, below=of state3] {$\cdots$};
    \node (output3) [block, above=of state3] {$\cdots$};
    \node (loss3) [block, above=of output3] {$\cdots$};
    \node (data3) [block, above=of loss3] {$\cdots$};

    \node (input4) [state, below=of state4] {$\x_\tau$};
    \node (output4) [state, above=of state4] {$\y_\tau$};
    \node (loss4) [function, above=of output4] {$L$};
    \node (data4) [data, above=of loss4] {$\ETA_\tau$};

    % Labels.
    \foreach \l in {state, input, output, loss, data} {
        \node [right=of \l4] {\l};
    }

    % Arrows.
    \foreach\i in {1, ..., 4} {
        \draw [path] (input\i) -- node [label, pos=0.4, right] {$\f$} (state\i);
        \draw [path] (state\i) -- node [label, pos=0.4, right] {$\h$} (output\i);
        \draw [path] (output\i) -- (loss\i);
        \draw [path] (data\i) -- (loss\i);
    }

    \foreach\i/\j in {1/2, 2/3, 3/4} {
        \draw [path] (data\i.-45) -- node [label, pos=0.45, right] {$\f$} (state\j.135);
    }
\end{tikzpicture}
%%% Local Variables:
%%% mode: latex
%%% TeX-master: "../rnn"
%%% End:

        \end{column}
        \begin{column}{0.53\textwidth}
            \begin{itemize}[<.->]
                \item Training: use data $\ETA_{t-1}$ for $\s_t$
                \item Forward \& backprop parallelizable
                \item<+-> Inference: connect output $\y_t$ to $\s_t$
                \item Can be combined with state-to-state or output-to-state connections for better performance
            \end{itemize}
        \end{column}
    \end{columns}
\end{frame}

\begin{frame}{Long short-term memory, figure}
    \centering
    \begin{tikzpicture}[node distance=1cm, x=1.12cm, y=9mm, auto]
    \clip (-4.8, 3.7) rectangle (4.8, -5.2);

    % Cells.
    \draw [cell] (-7.1, 2.5) rectangle (-3.4, -4.0);
    \draw [cell] (-2, 2.5) rectangle (1.7, -4.0);
    \draw [cell] (3.1, 2.5) rectangle (6.8, -4.0);

    % Nodes.
    \node (input) [coordinate] at (0, -4.6) {};
    \node (prev output) [coordinate] at (-4.1, 3.5) {};

    \node (plus) [plus] at (0, 0) {};

    \node (forget prod) [prod] at (-1, 0) {};
    \node (forget sigmoid) [gate] at (-1, -1.9) {$\sigma$};
    \node (forget affine) [gate] at (-1, -3.05) {$\mathrm{aff}$};

    \node (input prod) [prod] at (0, -0.9) {};
    \node (input sigmoid) [gate] at (1, -1.9) {$\sigma$};
    \node (input affine) [gate] at (1, -3.05) {$\mathrm{aff}$};
    \node (input tanh) [unary] at (0, -1.9) {$\tanh$};
    \node (input tanh affine) [unary] at (0, -3.05) {$\mathrm{aff}$};

    \node (state) [coordinate] at (1, 0) {};

    \node (output tanh) [unary] at (1, 0.9) {$\tanh$};
    \node (output prod) [prod] at (1, 2) {};
    \node (output sigmoid) [gate] at (0, 2) {$\sigma$};
    \node (output affine) [gate] at (-1, 2) {$\mathrm{aff}$};
    \node (output) [coordinate] at (1, 3.5) {};

    % Paths.

    % Inputs.
    \draw [path] (-4.1, 2.5) -- node [left, pos=0.8] {$\y_{t-1}$} (prev output);
    \draw [thick] (-4.1, 2.8) -| (-2.9, 0.075);
    \draw [thick] (-2.9, -0.075) |- (-0.7, -4.4) -- (-0.7, -3.85);
    \draw [thick] (input) -- node [right, pos=0.2] {$\x_t$} (0, -3.85);
    \draw [path] (-3.4, 0) -- node [pos=0.46, below] {$\s_{t-1}$} (forget prod);

    % Internal stuff.
    \draw [path] (forget affine) -- (forget sigmoid);
    \draw [path] (forget sigmoid) -- node [right, pos=0.4] {$\f_t$} (forget prod);
    \draw [path] (forget prod) -- (plus);

    \draw [path] (input affine) -- (input sigmoid);
    \draw [path] (input sigmoid) |- node [right, pos=0.3] {$\g_t$} (input prod);
    \draw [path] (input prod) -- (plus);

    \draw [path] (input tanh affine) -- (input tanh);
    \draw [path] (input tanh) -- (input prod);

    \draw [path] (plus) -- (3.1, 0);
    \draw [path] (state) -- node [pos=0, below] {$\s_t$} (output tanh);

    \draw [path] (output tanh) -- (output prod);
    \draw [path] (output affine) -- (output sigmoid);
    \draw [path] (output sigmoid) -- node [pos=0.4, below] {$\h_t$} (output prod);

    % Input to affine paths.
    \draw [ultra thick] (-1.7, -3.85) -- (1, -3.85);
    \draw [path] (-1, -3.85) -- (forget affine);
    \draw [path] (0, -3.85) -- (input tanh affine);
    \draw [path] (1, -3.85) -- (input affine);
    \draw [thick] (-1.7, -3.85) -- (-1.7, -0.075);
    \draw [path] (-1.7, 0.075) |- (output affine);

    % Outputs.
    \draw [path] (output prod) -- node [left, pos=0.8] {$\y_t$} (output);
    \draw [thick] (1, 2.8) -| (2.2, 0.075);
    \draw [path] (2.2, -0.075) |- (4.4, -4.4) -- (4.4, -4);
\end{tikzpicture}

%%% Local Variables:
%%% mode: latex
%%% TeX-master: "../rnn"
%%% End:

\end{frame}

\begin{frame}{Long short-term memory, equations}
    \begin{itemize}
        \item<+-> Inputs: $\x_t \in \Reals^q$; cell state: $\s_t \in \Reals^n$; output: $\y_t \in (-1, 1)^n$
        \item<.-> Weights and biases: $\U \in \Reals^{n \times q}$, $\W \in \Reals^{n \times n}$, $\b \in \Reals^n$
        \item<.-> $\sigma(x) = 1 / (1 + e^{-x})$, element-wise
        \item<.-> $\circ$: element-wise product
        \item<+-> Gates:
        \begin{align*}
            \text{forget:} \quad \f_t &= \sigma(\U_\f \x_t + \W_\f \y_{t-1} + \b_\f) \in (0, 1)^n \\
            \text{input:} \quad \g_t &= \sigma(\U_\g \x_t + \W_\g \y_{t-1} + \b_\g) \in (0, 1)^n \\
            \text{output:} \quad \h_t &= \sigma(\U_\h \x_t + \W_\h \y_{t-1} + \b_\h) \in (0, 1)^n
        \end{align*}
        \item<+-> Dynamics:
        \begin{align*}
            \text{state update:} \quad \s_t &= \f_t \circ \s_{t-1} + \g_t \circ \tanh(\U_\s \x_t + \W_\s \y_{t-1} + \b_\s) \\
            \text{output:} \quad \y_t &= \h_t \circ \tanh(\s_t)
        \end{align*}
        \item<.-> Total: $4 n (n + q + 1)$ parameters
    \end{itemize}
\end{frame}

\begin{frame}{Long short-term memory, comments}
    \begin{block}{}
        \vspace{-5mm}
        \begin{align*}
            \text{state update:} \quad \s_t &= \alert<2->{\f_t} \circ \s_{t-1} + \g_t \circ \tanh(\U_\s \x_t + \W_\s \y_{t-1} + \b_\s) \\
            \text{output:} \quad \y_t &= \h_t \circ \tanh(\s_t)
        \end{align*}
    \end{block}

    \begin{itemize}
        \item<+-> Introduced by~\citet{HochreiterNC97}; analysis by \citet{GreffIEEENNLS17}
        \item<.-> RNN{} problem: long backprop through time leads to vanishing/exploding gradients
        \item<+-> Key insight: \alert{forget gate $\f_t$} reduces dependency on states far in the past $\implies$ better-behaved gradients
        \item<.-> Gating behavior determined by inputs/outputs, not statically
        \item<+-> Typically initialize forget bias to $\b_\f = \one$~\citep{GersNC00,JozefowiczICML15}
        \begin{itemize}
            \item Otherwise, LSTM{} will forget too aggressively early in training
        \end{itemize}
        \item<.-> Can include state $\s_t$ in gate operands (peephole connections)
        \item<+-> The most common and arguably the strongest RNN{} architecture in use today \citep{JozefowiczICML15}
    \end{itemize}
\end{frame}

\begin{frame}{Gated recurrent unit, figure}
    \centering
    \begin{tikzpicture}[node distance=1cm, x=6.7mm, y=9mm, auto]
    \clip (-6, -0.2) rectangle (10, 8);

    % Cells.
    \draw [cell] (-12, 0.3) rectangle (-4, 7.5);
    \draw [cell] (-2.5, 0.3) rectangle (6.5, 7.5);
    \draw [cell] (8, 0.3) rectangle (15, 7.5);

    % Nodes.
    \node (input) [coordinate] at (0.3, -0.2) {};

    \node (update affine) [gate] at (-1, 3.5) {$\mathrm{aff}$};
    \node (update sigmoid) [gate] at (-1, 5) {$\sigma$};

    \node (reset affine) [gate] at (0.8, 3.5) {$\mathrm{aff}$};
    \node (reset sigmoid) [gate] at (2.5, 3.5) {$\sigma$};

    \node (state prod) [prod] at (-1, 7) {};
    \node (update prod) [prod] at (2.5, 1.3) {};

    \node (prev state) [coordinate] at (-4, 7) {};
    \node (prev state 0) [coordinate] at (-2, 7) {};
    \node (prev state 1) [coordinate] at (-2, 1.3) {};

    \node (subtract) [gate] at (2.25, 6) {$1 - \z_t$};
    \node (alt prod) [prod] at (5.5, 6) {};

    \node (alt affine) [unary] at (5.5, 2.67) {$\mathrm{aff}$};
    \node (alt tanh) [unary] at (5.5, 4.33) {$\tanh$};

    \node (sum) [plus] at (5.5, 7) {};
    \node (state) [coordinate] at (8, 7) {};

    % Paths.
    \draw [path] (input) -- node [right, pos=0.05] {$\x_t$} (0.3, 2.2);

    \draw [ultra thick] (-1.02, 2.2) -- (0.82, 2.2);
    \draw [path] (-1, 2.2) -- (update affine);
    \draw [path] (0.8, 2.2) -- (reset affine);

    \draw [path] (update affine) -- (update sigmoid);
    \draw [path] (reset affine) -- (reset sigmoid);

    \draw [path] (prev state) -- node [below, pos=0.27] {$\s_{t-1}$} (state prod);
    \draw [path] (-3.5, 7) -- (-3.5, 8);
    \draw [path] (update sigmoid) -- (state prod);
    \draw [path] (-1, 6) -- node [left, pos=0] {$\z_t$}(subtract);
    \draw [path] (subtract) -- (alt prod);
    \draw [path] (state prod) -- (sum);
    \draw [path] (reset sigmoid) -- node [right, pos=0.4] {$\r_t$} (update prod);

    \draw [thick] (prev state 0) -- (prev state 1);
    \draw [thick] (prev state 1) -- (0.2, 1.3);
    \draw [path] (0.4, 1.3) -- (update prod);
    \draw [path] (-0.5, 1.3) -- (-0.5, 2.2);

    \draw [ultra thick] (4, 1.32) -- (4, 0.68);
    \draw [path] (update prod) -- (4, 1.3);
    \draw [path] (0.3, 0.7) -- (4, 0.7);
    \draw [path] (4, 1) -| (alt affine);
    \draw [path] (alt affine) -- (alt tanh);
    \draw [path] (alt tanh) -- node [right, pos=0.4] {$\st_t$} (alt prod);
    \draw [path] (alt prod) -- (sum);

    \draw [path] (sum) -- (state);
    \draw [path] (7, 7) -- node [below, pos=0] {$\s_t$}(7, 8);
\end{tikzpicture}

%%% Local Variables:
%%% mode: latex
%%% TeX-master: "../rnn"
%%% End:

\end{frame}

\begin{frame}{Gated recurrent unit, equations}
    \begin{itemize}
        \item<+-> Inputs: $\x_t \in \Reals^q$; cell state and output: $\s_t \in (-1, 1)^n$
        \item<.-> Weights and biases: $\U \in \Reals^{n \times q}$, $\W \in \Reals^{n \times n}$, $\b \in \Reals^n$
        \item<+-> Gates:
        \begin{align*}
            \text{reset:} \quad \r_t &= \sigma(\U_\r \x_t + \W_\r \s_{t-1} + \b_\r) \in (0, 1)^n \\
            \text{update:} \quad \z_t &= \sigma(\U_\z \x_t + \W_\z \s_{t-1} + \b_\z) \in (0, 1)^n
        \end{align*}
        \item<+-> Dynamics:
        \begin{align*}
            \text{alternate state:} \quad \st_t &= \tanh(\U_\s \x_t + \W_\s (\r_t \circ \s_{t-1}) + \b_\s) \in (-1, 1)^n \\
            \text{state update:} \quad \s_t &= \z_t \circ \s_{t-1} + (\one - \z_t) \circ \st_t
        \end{align*}
        \item<.-> Total: $3 n (n + q + 1)$ parameters
    \end{itemize}
\end{frame}

\begin{frame}{Gate recurrent unit, comments}
    \begin{block}{}
        \vspace{-5mm}
        \begin{align*}
            \text{alternate state:} \quad \st_t &= \tanh(\U_\s \x_t + \W_\s (\alert{\r_t} \circ \s_{t-1}) + \b_\s) \in (-1, 1)^n \\
            \text{state update:} \quad \s_t &= \alert{\z_t} \circ \s_{t-1} + \alert{(\one - \z_t)} \circ \st_t
        \end{align*}
    \end{block}

    \begin{itemize}
        \item<+-> Introduced by \citet{ChoEMNLP14}
        \item<.-> \alert{Reset gate $\r_t$} limits how much $\s_{t-1}$ plays into $\st_t$
        \item<.-> \alert{Update gate $\z_t$} similar to LSTM{} forget gate, but linearly interpolates between $\s_{t-1}$ and $\st_t$
        \item<+-> Hand-wavy argument: LSTM{} output \& output gate not really necessary
        \begin{itemize}
            \item Too flexible, not enough structure, too many parameters $\implies$ harder to train
            \item Recall: RNN{}s preferable to dense networks because imposing structure makes good network easier to obtain via optimization
        \end{itemize}
        \item<+-> Also arguably the strongest RNN{} architecture today \citep{JozefowiczICML15}
        \begin{itemize}
            \item No consensus on LSTM{} vs.~GRU{}
        \end{itemize}
    \end{itemize}
\end{frame}

\begin{frame}{Bidirectionality}
    \begin{columns}
        \begin{column}{0.53\textwidth}
            \begin{tikzpicture}[node distance=1cm]
    % Reverse LSTM cells.
    \node (reverse 1) [lstm, fill=green!20] {RNN};
    \node (reverse 2) [lstm, fill=green!20, right=of reverse 1] {RNN};
    \node (reverse 3) [block, right=of reverse 2] {$\cdots$};
    \node (reverse 4) [lstm, fill=green!20, right=of reverse 3] {RNN};

    % Forward LSTM cells.
    \foreach \x in {1, 2, 4} {%
        \node (forward \x) [lstm, below=7mm of reverse \x, xshift=1cm] {RNN};
    }

    \node (forward 3) [block, below=7mm of reverse 3, xshift=1cm] {$\cdots$};

    % Labels.
    \foreach \j in {1, 2} {%
        \node (input \j) [below=of forward \j] {$\x_\j$};
        \node (output \j) [above=of reverse \j, xshift=1.5mm] {$\y_\j$};
    }

    \node (input 3) [below=of forward 3] {$\cdots$};
    \node (input 4) [below=of forward 4] {$\x_n$};
    \node (output 3) [above=of reverse 3, xshift=1.5mm] {$\cdots$};
    \node (output 4) [above=of reverse 4, xshift=1.5mm] {$\y_n$};

    % Coordinates.
    \foreach \j in {1, ..., 4} {%
        \node (i \j) [coordinate, below=6mm of forward \j] {};
        % Corner coordinate.
        \node (y \j) [coordinate, above=4mm of reverse \j, xshift=3mm] {};
        % Two output coordinates.
        \node (forward y \j) [coordinate, above=6mm of y \j] {};
        \node (reverse y \j) [coordinate, above=of reverse \j] {};
    }

    % Coordinates around the reverse state flow.
    \foreach \j in {1, ..., 3} {%
        \node (reverse state down \j) [coordinate, above=10mm of forward \j] {};
        \node (reverse state up \j) [coordinate, above=1.25mm of reverse state down \j] {};
    }

    % Coordinates around the forward state flow.
    \foreach \j in {2, ..., 4} {%
        \node (forward state up \j) [coordinate, below=10mm of reverse \j] {};
        \node (forward state down \j) [coordinate, below=1.25mm of forward state up \j] {};
    }

    % Paths.

    % Cell to cell flow.
    \foreach \j in {1, ..., 3} {
        \pgfmathtruncatemacro{\k}{\j + 1}

        \draw [path] (forward \j) -- (forward \k);
        \draw [path] (reverse \k) -- (reverse \j);
    }

    % Input to forward LSTM and reverse LSTM to output.
    \foreach \j in {1, ..., 4} {%
        \draw [path] (input \j) -- (forward \j);
        \draw [path] (reverse \j) -- (reverse y \j);
    }

    % Secondary input/output paths with no path intersections.
    \draw [path] (i 1) -| (reverse 1);
    \draw [path] (forward 4) |- (y 4) -- (forward y 4);

    % Forward LSTM outputs.
    \foreach \j in {1, ..., 3} {%
        \draw [thick] (forward \j) -- (reverse state down \j);
        \draw [path] (reverse state up \j) |- (y \j) -- (forward y \j);
    }

    % Reverse LSTM inputs.
    \foreach \j in {2, ..., 4} {%
        \draw [thick] (i \j) -| (forward state down \j);
        \draw [path] (forward state up \j) -- (reverse \j);
    }
\end{tikzpicture}

%%% Local Variables:
%%% mode: latex
%%% TeX-master: "../rnn"
%%% End:

        \end{column}
        \begin{column}{0.47\textwidth}
            \begin{itemize}[<.->]
                \item<+-> Humans tend to process sequences begin-to-end
                \item Often, RNN{}s need not do the same
                \item E.g., English word generation: \texttt{\ldots{}ing} likely followed by \texttt{<end>}, but also \texttt{ng<end>} likely preceded by \texttt{\ldots{}i}
            \end{itemize}
            \uncover<+->{Bidirectional RNN{}: \textcolor{blue}{forward}- and \textcolor{Green4}{reverse}-directional RNN{}s with separate states, outputs, weights, biases}
            \begin{itemize}[<.->]
                \item Each receives same inputs
                \item Outputs concatenated at each step
                \item Easy to implement---1 line of code
                \item \citet{SchusterIEEESP97}
            \end{itemize}
        \end{column}
    \end{columns}
\end{frame}

\begin{frame}{Stacked RNN{}s}
    \begin{columns}
        \begin{column}{0.41\textwidth}
            \begin{tikzpicture}[node distance=6mm]
    % Nodes.

    % Inputs.
    \node (node 01) [block] {$\x_1$};
    \node (node 02) [block, right=of node 01] {$\x_2$};
    \node (node 03) [block, right=of node 02] {$\cdots$};
    \node (node 04) [block, right=of node 03] {$\x_\tau$};

    % RNNs.
    \foreach \i in {1, 2, 4} {%
        \node (node 1\i) [lstm, above=of node 0\i] {RNN};
        \node (node 2\i) [lstm, above=of node 1\i] {RNN};
        \node (node 3\i) [block, above=of node 2\i] {\vvdots};
        \node (node 4\i) [lstm, above=of node 3\i] {RNN};
        \node (node 5\i) [block, above=of node 4\i] {$\y_\i$};
    }

    % Ellipses.
    \node (node 13) [block, above=of node 03] {$\cdots$};
    \node (node 23) [block, above=of node 13] {$\cdots$};
    \node (node 33) [block, above=of node 23] {\vvdots};
    \node (node 43) [block, above=of node 33] {$\cdots$};
    \node (node 53) [block, above=of node 43] {$\cdots$};

    % Horizontal arrows.
    \foreach \n in {1, 2, 4} {%
        \foreach \i in {1, ..., 3} {
            \pgfmathtruncatemacro{\j}{\i + 1}
            \draw [path] (node \n\i) -- (node \n\j);
        }
    }

    % Vertical arrows.
    \foreach \n in {0, ..., 4} {
        \pgfmathtruncatemacro{\m}{\n + 1}

        \foreach \i in {1, ..., 4} {%
            \draw [path] (node \n\i) -- (node \m\i);
        }
    }
\end{tikzpicture}%
%%% Local Variables:
%%% mode: latex
%%% TeX-master: "../rnn"
%%% End:

        \end{column}
        \begin{column}{0.59\textwidth}
            \begin{itemize}
                \item LSTM{}s and GRU{}s are powerful, but they're still shallow
                \item By far the most common way to make them deep is to \alert{stack} them
                \item Output of one RNN{} used as input to another; repeat \emph{ad infinitum}
                \pause
                \item RNN{}s can be bidirectional
                \item Easy to implement---1 or 2 lines of code
                \item Common to use same state size in all layers
                \begin{itemize}
                    \item I doubt there's much rigorous justification
                \end{itemize}
            \end{itemize}
        \end{column}
    \end{columns}
\end{frame}

\begin{frame}{Other deep RNN{}s: deep input-to-state connections}
    \hspace{2cm}
    \begin{tikzpicture}[node distance=5mm, auto]
    % States.
    \node (state11) [state] {$\s_{11}$};
    \node (state12) [state, right=7mm of state11] {$\s_{12}$};
    \node (state13) [block, right=7mm of state12] {$\cdots$};
    \node (state14) [state, right=7mm of state13] {$\s_{1\tau}$};

    \node (state21) [state, above=of state11] {$\s_{21}$};
    \node (state20) [state, left=7mm of state21] {$\s_{20}$};
    \node (state22) [state, above=of state12] {$\s_{22}$};
    \node (state23) [block, above=of state13] {$\cdots$};
    \node (state24) [state, above=of state14] {$\s_{2\tau}$};


    % Inputs, outputs, losses.
    \foreach\i in {1,2} {
        \node (input\i) [state, below=of state1\i] {$\x_\i$};
        \node (output\i) [state, above=of state2\i] {$\y_\i$};
        \node (loss\i) [function, above=of output\i] {$L$};
        \node (data\i) [data, above=of loss\i] {$\ETA_\i$};
    }

    \node (input3) [block, below=of state13] {$\cdots$};
    \node (output3) [block, above=of state23] {$\cdots$};
    \node (loss3) [block, above=of output3] {$\cdots$};
    \node (data3) [block, above=of loss3] {$\cdots$};

    \node (input4) [state, below=of state14] {$\x_\tau$};
    \node (output4) [state, above=of state24] {$\y_\tau$};
    \node (loss4) [function, above=of output4] {$L$};
    \node (data4) [data, above=of loss4] {$\ETA_\tau$};

    % Labels.
    \foreach \l in {input, output, loss, data} {
        \node [right=7mm of \l4] {\l};
    }

    \node [right=7mm of state14] {hidden};
    \node [right=7mm of state24] {state 1};

    % Arrows.
    \foreach\i in {1, ..., 4} {
        \draw [path] (input\i) -- node [label, pos=0.4, right] {$\f$} (state1\i);
        \draw [path] (state1\i) -- node [label, pos=0.4, right] {$\g$} (state2\i);
        \draw [path] (state2\i) -- node [label, pos=0.4, right] {$\h$} (output\i);
        \draw [path] (output\i) -- (loss\i);
        \draw [path] (data\i) -- (loss\i);
    }

    \foreach\i/\j in {0/1, 1/2, 2/3, 3/4} {
        \draw [path] (state2\i) -- node [label, pos=0.4] {$\g$} (state2\j);
    }
\end{tikzpicture}
%%% Local Variables:
%%% mode: latex
%%% TeX-master: "../rnn"
%%% End:

    \begin{tikzpicture}[node distance=5mm, auto]
    \node (state 1) [state] {$\s_{1t}$};
    \node (state 2) [state, above=of state 1] {$\s_{2t}$};
    \node (input) [state, below=of state 1] {$\x_t$};
    \node (output) [state, above=of state 2] {$\y_t$};
    \node (loss) [function, above=of output] {$L$};
    \node (data) [data, above=of loss] {$\ETA_t$};

    \foreach \i in {1, 2} {%
        \node (skip \i) [inner sep=0.5mm, above=1mm of state \i] {};
        \node (skip \i l) [coordinate, left=7mm of skip \i] {};
        \node (skip \i r) [coordinate, right=7mm of skip \i] {};
    }

    % Arrows.
    \draw [path] (input) -- node [label, pos=0.4, right] {$\f$} (state 1);
    \draw [path] (state 1) -- node [label, pos=0.4, right] {$\g$} (state 2);
    \draw [path] (state 2) -- node [label, pos=0.7, right] {$\h$} (output);
    \draw [path] (output) -- (loss);
    \draw [path] (data) -- (loss);

    \foreach \i/\j in {2/\g} {%
        \draw [thick] (state \i) -| node [label, right, pos=0.7] {$\j$} (skip \i r) -- (skip \i);
        \draw [path] (skip \i) -- (skip \i l) |- (state \i);
    }
\end{tikzpicture}
%%% Local Variables:
%%% mode: latex
%%% TeX-master: "../rnn"
%%% End:

\end{frame}

\begin{frame}{Other deep RNN{}s: deep state-to-output connections}
    \hspace{2cm}
    \begin{tikzpicture}[node distance=5mm, auto]
    % States.
    \node (state11) [state] {$\s_{11}$};
    \node (state10) [state, left=7mm of state11] {$\s_{10}$};
    \node (state12) [state, right=7mm of state11] {$\s_{12}$};
    \node (state13) [block, right=7mm of state12] {$\cdots$};
    \node (state14) [state, right=7mm of state13] {$\s_{1\tau}$};

    \node (state21) [state, above=of state11] {$\s_{21}$};
    \node (state22) [state, above=of state12] {$\s_{22}$};
    \node (state23) [block, above=of state13] {$\cdots$};
    \node (state24) [state, above=of state14] {$\s_{2\tau}$};


    % Inputs, outputs, losses.
    \foreach\i in {1,2} {
        \node (input\i) [state, below=of state1\i] {$\x_\i$};
        \node (output\i) [state, above=of state2\i] {$\y_\i$};
        \node (loss\i) [function, above=of output\i] {$L$};
        \node (data\i) [data, above=of loss\i] {$\ETA_\i$};
    }

    \node (input3) [block, below=of state13] {$\cdots$};
    \node (output3) [block, above=of state23] {$\cdots$};
    \node (loss3) [block, above=of output3] {$\cdots$};
    \node (data3) [block, above=of loss3] {$\cdots$};

    \node (input4) [state, below=of state14] {$\x_\tau$};
    \node (output4) [state, above=of state24] {$\y_\tau$};
    \node (loss4) [function, above=of output4] {$L$};
    \node (data4) [data, above=of loss4] {$\ETA_\tau$};

    % Labels.
    \foreach \l in {input, output, loss, data} {
        \node [right=7mm of \l4] {\l};
    }

    \node [right=7mm of state14] {state 1};
    \node [right=7mm of state24] {hidden};

    % Arrows.
    \foreach\i in {1, ..., 4} {
        \draw [path] (input\i) -- node [label, pos=0.4, right] {$\f$} (state1\i);
        \draw [path] (state1\i) -- node [label, pos=0.4, right] {$\g$} (state2\i);
        \draw [path] (state2\i) -- node [label, pos=0.4, right] {$\h$} (output\i);
        \draw [path] (output\i) -- (loss\i);
        \draw [path] (data\i) -- (loss\i);
    }

    \foreach\i/\j in {0/1, 1/2, 2/3, 3/4} {
        \draw [path] (state1\i) -- node [label, pos=0.4] {$\f$} (state1\j);
    }
\end{tikzpicture}
%%% Local Variables:
%%% mode: latex
%%% TeX-master: "../rnn"
%%% End:

    \begin{tikzpicture}[node distance=5mm, auto]
    \node (state 1) [state] {$\s_{1t}$};
    \node (state 2) [state, above=of state 1] {$\s_{2t}$};
    \node (input) [state, below=of state 1] {$\x_t$};
    \node (output) [state, above=of state 2] {$\y_t$};
    \node (loss) [function, above=of output] {$L$};
    \node (data) [data, above=of loss] {$\ETA_t$};

    \foreach \i in {1, 2} {%
        \node (skip \i) [inner sep=0.5mm, above=1mm of state \i] {};
        \node (skip \i l) [coordinate, left=7mm of skip \i] {};
        \node (skip \i r) [coordinate, right=7mm of skip \i] {};
    }

    % Arrows.
    \draw [path] (input) -- node [label, pos=0.4, right] {$\f$} (state 1);
    \draw [path] (state 1) -- node [label, pos=0.7, right] {$\g$} (state 2);
    \draw [path] (state 2) -- node [label, pos=0.4, right] {$\h$} (output);
    \draw [path] (output) -- (loss);
    \draw [path] (data) -- (loss);

    \foreach \i/\j in {1/\f} {%
        \draw [thick] (state \i) -| node [label, right, pos=0.7] {$\j$} (skip \i r) -- (skip \i);
        \draw [path] (skip \i) -- (skip \i l) |- (state \i);
    }
\end{tikzpicture}
%%% Local Variables:
%%% mode: latex
%%% TeX-master: "../rnn"
%%% End:

\end{frame}

\begin{frame}{Other deep RNN{}s: multiple recurrent state layers \citep{GravesICASSP13}}
    \hspace{2cm}
    \begin{tikzpicture}[node distance=5mm, auto]
    % States.
    \node (state11) [state] {$\s_{11}$};
    \node (state10) [state, left=7mm of state11] {$\s_{10}$};
    \node (state12) [state, right=7mm of state11] {$\s_{12}$};
    \node (state13) [block, right=7mm of state12] {$\cdots$};
    \node (state14) [state, right=7mm of state13] {$\s_{1\tau}$};

    \node (state21) [state, above=of state11] {$\s_{21}$};
    \node (state20) [state, left=7mm of state21] {$\s_{20}$};
    \node (state22) [state, above=of state12] {$\s_{22}$};
    \node (state23) [block, above=of state13] {$\cdots$};
    \node (state24) [state, above=of state14] {$\s_{2\tau}$};


    % Inputs, outputs, losses.
    \foreach\i in {1,2} {
        \node (input\i) [state, below=of state1\i] {$\x_\i$};
        \node (output\i) [state, above=of state2\i] {$\y_\i$};
        \node (loss\i) [function, above=of output\i] {$L$};
        \node (data\i) [data, above=of loss\i] {$\ETA_\i$};
    }

    \node (input3) [block, below=of state13] {$\cdots$};
    \node (output3) [block, above=of state23] {$\cdots$};
    \node (loss3) [block, above=of output3] {$\cdots$};
    \node (data3) [block, above=of loss3] {$\cdots$};

    \node (input4) [state, below=of state14] {$\x_\tau$};
    \node (output4) [state, above=of state24] {$\y_\tau$};
    \node (loss4) [function, above=of output4] {$L$};
    \node (data4) [data, above=of loss4] {$\ETA_\tau$};

    % Labels.
    \foreach \l in {input, output, loss, data} {
        \node [right=7mm of \l4] {\l};
    }

    \foreach \i in {1, 2} {%
        \node [right=7mm of state\i4] {state \i};
    }

    % Arrows.
    \foreach\i in {1, ..., 4} {
        \draw [path] (input\i) -- node [label, pos=0.4, right] {$\f$} (state1\i);
        \draw [path] (state1\i) -- node [label, pos=0.4, right] {$\g$} (state2\i);
        \draw [path] (state2\i) -- node [label, pos=0.4, right] {$\h$} (output\i);
        \draw [path] (output\i) -- (loss\i);
        \draw [path] (data\i) -- (loss\i);
    }

    \foreach\i/\j in {0/1, 1/2, 2/3, 3/4} {
        \draw [path] (state1\i) -- node [label, pos=0.4] {$\f$} (state1\j);
        \draw [path] (state2\i) -- node [label, pos=0.4] {$\g$} (state2\j);
    }
\end{tikzpicture}
%%% Local Variables:
%%% mode: latex
%%% TeX-master: "../rnn"
%%% End:

    \begin{tikzpicture}[node distance=5mm, auto]
    \node (state 1) [state] {$\s_{1t}$};
    \node (state 2) [state, above=of state 1] {$\s_{2t}$};
    \node (input) [state, below=of state 1] {$\x_t$};
    \node (output) [state, above=of state 2] {$\y_t$};
    \node (loss) [function, above=of output] {$L$};
    \node (data) [data, above=of loss] {$\ETA_t$};

    \foreach \i in {1, 2} {%
        \node (skip \i) [inner sep=0.5mm, above=1mm of state \i] {};
        \node (skip \i l) [coordinate, left=7mm of skip \i] {};
        \node (skip \i r) [coordinate, right=7mm of skip \i] {};
    }

    % Arrows.
    \draw [path] (input) -- node [label, pos=0.4, right] {$\f$} (state 1);
    \draw [path] (state 1) -- node [label, pos=0.7, right] {$\g$} (state 2);
    \draw [path] (state 2) -- node [label, pos=0.7, right] {$\h$} (output);
    \draw [path] (output) -- (loss);
    \draw [path] (data) -- (loss);

    \foreach \i/\j in {1/\f, 2/\g} {%
        \draw [thick] (state \i) -| node [label, right, pos=0.7] {$\j$} (skip \i r) -- (skip \i);
        \draw [path] (skip \i) -- (skip \i l) |- (state \i);
    }
\end{tikzpicture}
%%% Local Variables:
%%% mode: latex
%%% TeX-master: "../rnn"
%%% End:

\end{frame}

\begin{frame}{Other deep RNN{}s: deep state-to-state connections \citep{PascanuICLR14}}
    \begin{tikzpicture}[node distance=7mm, auto]
    % States.
    \node (state05) [state] {$\z_0$};
    \node (state1) [state, right=5mm of state05] {$\s_1$};
    \node (state15) [state, right=5mm of state1] {$\z_1$};
    \node (state2) [state, right=5mm of state15] {$\s_2$};
    \node (state25) [state, right=5mm of state2] {$\z_2$};
    \node (state3) [block, right=5mm of state25] {$\cdots$};
    \node (state35) [state, right=5mm of state3] {$\z_{\tau-1}$};
    \node (state4) [state, right=5mm of state35] {$\s_\tau$};

    % Inputs, outputs, losses.
    \foreach \i in {1,2} {
        \node (input\i) [state, below=of state\i] {$\x_\i$};
        \node (output\i) [state, above=of state\i] {$\y_\i$};
        \node (loss\i) [function, above=of output\i] {$L$};
        \node (data\i) [data, above=of loss\i] {$\ETA_\i$};
    }

    \node (input3) [block, below=of state3] {$\cdots$};
    \node (output3) [block, above=of state3] {$\cdots$};
    \node (loss3) [block, above=of output3] {$\cdots$};
    \node (data3) [block, above=of loss3] {$\cdots$};

    \node (input4) [state, below=of state4] {$\x_\tau$};
    \node (output4) [state, above=of state4] {$\y_\tau$};
    \node (loss4) [function, above=of output4] {$L$};
    \node (data4) [data, above=of loss4] {$\ETA_\tau$};

    % Skip connection points.
    \foreach \i in {1, ..., 3} {%
        \node (control\i1) [coordinate, below=3mm of state\i5.225] {};
        \node (control\i2) [coordinate, below=3mm of state\i5.315] {};
    }

    % Labels.
    \foreach \l in {state, input, output, loss, data} {
        \node [right=of \l4] {\l};
    }

    % Arrows.
    \foreach \i in {1, ..., 4} {
        \draw [path] (input\i) -- node [label, pos=0.4, right] {$\f$} (state\i);
        \draw [path] (state\i) -- node [label, pos=0.4, right] {$\h$} (output\i);
        \draw [path] (output\i) -- (loss\i);
        \draw [path] (data\i) -- (loss\i);
    }

    \foreach \i in {1, ..., 3} {%
        \draw [path] (state\i) -- node [label, pos=0.4] {$\g$} (state\i5);
    }

    \foreach \i in {0, ..., 3} {
        \pgfmathtruncatemacro{\j}{\i + 1}
        \draw [path] (state\i5) -- node [label, pos=0.4] {$\f$} (state\j);
    }

    % Skip connections.
    \uncover<2->{
        \foreach \i in {1, ..., 3} {
            \pgfmathtruncatemacro{\j}{\i + 1}
            \draw [path] (state\i) ..
            controls (control\i1) and (control\i2) ..
            node [label, below] {$\f$} (state\j);
        }
    }
\end{tikzpicture}
%%% Local Variables:
%%% mode: latex
%%% TeX-master: "../rnn"
%%% End:

    \begin{tikzpicture}[node distance=7mm, auto]
    \node (state) [state] {$\s_t$};
    \node (state 1) [state, right=5mm of state] {$\z_t$};
    \node (input) [state, below=of state] {$\x_t$};
    \node (output) [state, above=of state] {$\y_t$};
    \node (loss) [function, above=of output] {$L$};
    \node (data) [data, above=of loss] {$\ETA_t$};

    \node (skip 0) [inner sep=0.5mm, above=1mm of state] {};
    \node (skip 00) [coordinate, left=8mm of skip 0] {};
    \node (skip 01) [coordinate, right=19mm of skip 0] {};

    \node (skip 1) [inner sep=0.5mm, below=2.5mm of state] {};
    \node (skip 10) [coordinate, left=8mm of skip 1] {};
    \node (skip 11) [coordinate, right=6mm of skip 1] {};

    % Arrows.
    \draw [path] (input) --
    node [label, pos=0.25, right, xshift=-0.5mm] {$\f$}
    (state);
    \draw [path] (state) -- node [label, pos=0.6, right] {$\h$} (output);
    \draw [path] (output) -- (loss);
    \draw [path] (data) -- (loss);

    \draw [path] (state.15) --
    node [label, pos=0.4, yshift=-0.5mm] {$\g$}
    (state 1.165);
    \draw [thick] (state 1.15) -|
    node [label, right, pos=0.7, xshift=-0.5mm] {$\f$}
    (skip 01) -- (skip 0);
    \draw [path] (skip 0) -- (skip 00) |- (state.165);

    \uncover<2->{
        \draw [thick] (state.345) -|
        node [label, pos=0.9, xshift=-0.5mm] {$\f$}
        (skip 11) -- (skip 1);
        \draw [path] (skip 1) -- (skip 10) |- (state.190);
    }
\end{tikzpicture}
%%% Local Variables:
%%% mode: latex
%%% TeX-master: "../rnn"
%%% End:

\end{frame}

\begin{frame}{Odds \& ends (1/2)}
    \begin{itemize}
        \item<+-> Skip connections help prevent vanishing/exploding gradients by shortening state-to-state paths
        \item<+-> Gradient clipping helps prevent exploding gradients \citep{MikolovPhD12,PascanuICML13}
        \item<+-> Stateful RNN{}
        \begin{itemize}[<.->]
            \item Typically, initialize cell state to $\s_0 = \zero$
            \item<+-> Stateful: set $\s_0$ to be $\s_\tau$ from the last RNN{} invocation
            \item Used when actual sequence length $>$ RNN length: each RNN invocation extends previous one
            \item Use truncated backpropagation through time: only backprop through beginning of RNN, not beginning of sequence
        \end{itemize}
    \end{itemize}

    \centering
    \begin{tikzpicture}[x=1.15cm, y=5mm]
    \node (input 0) [coordinate] at (0, 0) {};

    \foreach \i in {0, ..., 8} {
        \node (input \i) [fill=black, circle, minimum width=1.4mm, inner sep=0pt] at (\i, 0) {};
    }

    \uncover<2>{
        \foreach \i in {0, ..., 2} {
            \node (rnn \i) [lstm, above=5mm of input \i, font=\footnotesize] {RNN};
        }
    }

    \node (rnn -1) [left=3.5mm of rnn 0, inner sep=1pt] {$\zero$};

    \uncover<3>{
        \foreach \i in {3, ..., 5} {
            \node (rnn \i) [lstm, above=5mm of input \i, font=\footnotesize] {RNN};
        }
    }

    \uncover<4>{
        \foreach \i in {6, ..., 8} {
            \node (rnn \i) [lstm, above=5mm of input \i, font=\footnotesize] {RNN};
        }
    }

    \foreach \i in {0, ..., 8} {
        \node (output \i) [coordinate, above=5mm of rnn \i] {};
    }

    \draw [path] (rnn -1) -- (rnn 0);

    \uncover<2>{
        \draw [path] (input 0) -- (rnn 0);
        \draw [path] (rnn 0) -- (output 0);

        \foreach \i in {1, 2} {
            \pgfmathtruncatemacro{\j}{\i - 1}
            \draw [path] (input \i) -- (rnn \i);
            \draw [path] (rnn \j) -- (rnn \i);
            \draw [path] (rnn \i) -- (output \i);
        }
    }

    \uncover<3>{
        \foreach \i in {3, ..., 5} {
            \pgfmathtruncatemacro{\j}{\i - 1}
            \draw [path] (input \i) -- (rnn \i);
            \draw [path] (rnn \j) -- (rnn \i);
            \draw [path] (rnn \i) -- (output \i);
        }
    }

    \uncover<4>{
        \foreach \i in {6, ..., 8} {
            \pgfmathtruncatemacro{\j}{\i - 1}
            \draw [path] (input \i) -- (rnn \i);
            \draw [path] (rnn \j) -- (rnn \i);
            \draw [path] (rnn \i) -- (output \i);
        }
    }
\end{tikzpicture}

%%% Local Variables:
%%% mode: latex
%%% TeX-master: "../rnn"
%%% End:

\end{frame}

\begin{frame}{Odds \& ends (2/2)}
    \begin{itemize}
        \item<+-> Variable-length RNN{}
        \begin{itemize}
            \item Beauty of RNN{} is that length $\tau$ need not be fixed
            \item Recall silly example: variable length of speech auto $\to$ variable length sentence
            \item But static-length RNN{}s are more computationally optimizable
        \end{itemize}
        \item<+-> Variable-length inputs/outputs
        \begin{itemize}
            \item Common approach: add \texttt{<end>} token to vocabulary; stop RNN{} when \texttt{<end>} sampled
            \item More generally: add extra scalar output that ends RNN{} when condition met
            \item Alternate approach: add output that predicts length $\tau$; then run for $\tau$ steps
            \begin{itemize}
                \item Important to add $\tau$ or $\tau - t$ as input to node $t$, lest sequence ends abruptly
            \end{itemize}
        \end{itemize}
    \end{itemize}
\end{frame}

%%% Local Variables:
%%% mode: latex
%%% TeX-master: "../rnn"
%%% End:
